%第一块代码:强制位置,不能随便移动的图片。
\begin{figure}[H]
    \centering
    \includegraphics[scale=0.6]{standard.PNG}
    \caption{evaluation criterion}
    \label{1}
\end{figure}

%第二块代码:大括号,左对齐,有标号
\begin{equation}
    s.t\left\{
    \begin{array}{lr}
    A_{j total}\le \sum_k^2 l_k\times N_k\times n_k&k\in {sc,dc}\\
    CI_j=\frac{d_{j total}}{n_{dc}\times N_{dc}+n_{sc}\times N_{sc}}\le 1&\\
    P_{eff}=\frac{n_{dc}}{n_{sc}}&\\
    n_{sc}\le \frac{N_j}{N_{sc}}&\\
    n_{dc},n_{sc}\ge 0(int)&\\
    \end{array}
    \right.
\end{equation}

%第三块代码:在文字前面加黑点



%第四块代码:导入图片
\begin{figure}[ht]
    \centering
    \includegraphics[scale=0.6]{standard.PNG}
    \caption{evaluation criterion}
    \label{1}
\end{figure}

%第五块代码:绘制表格
\begin{table}[ht]
\centering
\begin{tabular}{|c|c|c|c|c|c|}
\toprule
Year&Original value&Predictive value&Residual&Relative error&Level error\\
\midrule
2014&0.000130974&0.000130974&0&0&0\\
\midrule
2015&0.000222056&0.000168762&0.000053294&0.057798804&-0.116294912\\
\bottomrule
\end{tabular}
\caption{Grey prediction model data checklist}
\end{table}

%第六块代码:表格2
\begin{table}[!htbp]
\centering
\begin{tabular}{|c|c|c|}
\hline
\multicolumn{3}{|c|}{学生信息}\\ % 用\multicolumn{3}表示横向合并三列 
                        % |c|表示居中并且单元格两侧添加竖线 最后是文本
\hline
姓名&学号&性别\\
\hline
Jack& 001& Male\\
\hline
Angela& 002& Female\\
\hline
\end{tabular}
\caption{这是一张三线表}
\end{table}

%加数字
\begin{enumerate}
    \item We do ...
    \item We do ...
    \item We do ...
\end{enumerate}